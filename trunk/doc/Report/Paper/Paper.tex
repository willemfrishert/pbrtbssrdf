\documentclass{article}

\title{Rendering Translucent Objects using Subsurface Scattering}
\author{Jo\~{a}o Pedro Jorge \and Willem Frishert}
\date{29-01-2007}

\begin{document}
\maketitle

\section{Introduction}
What are translucent objects and subsurface scattering definitions.
\subsection{BRDF's vs. BSSRDF's}
Small comparison, providing both equations

\section{Model}
Introduce the model; mention that it comes from the previous paper and more information can be found in there;
\subsection{The Diffusion Approximation}
The idea: comes from medical sciences - the dipole; the importance of multiple scattering - dominant
\subsection{Two-Pass Technique for Evaluating the Diffusion Approximation}
Introduction to the two pass techinque: key idea, to make it faster decouple the computation of irradiance...
\subsubsection{Sampling the Irradiance}
Good to have uniform distribution of sample points - the authors used Turk's algorithm; Number of points defined by lu and total area; usage of GI techniques;
equations
\subsubsection{Evaluating the Diffusion Approximation}
Can be used directly - heavy, lots of points; speak about the 3 options; hierarchical evaluation: best of both worlds; storage on the tree; subdivision criterion; computation of the radiant exitance and finally computing the Lo for each point.
input parameters: sigmas

\section{Implementation}
divided work into two parts: pass1 and pass2; svn server to synch; 
\subsection{Renderer}
talk about Renderman; change to PBRT (reasons to change);
\subsection{First Pass - Sampling The Irradiance}
Turk's algorithm explanation; usage of basic MC estimator; storage of points on the octree - next section.

\subsection{Second Pass - Evaluating the Diffusion Approximation}
octree - add and lookup; one or two schemes; the epsilon factor;

\section{Results}
\section{Conclusion}

% ### BIBLIOGRAPHY ###
\begin{thebibliography}{99}

%
% WIKI REFERENCE
%

\bibitem{WelshParallax} Welsh, Terry, 2004, {\it Parallax Mapping with Offset Limiting: A PerPixel Approximation of Uneven Surfaces}

\bibitem{TatarchukParallax} Tatarchuk, Natalya, 2006, Dynamic Parallax Occlusion Mapping with Approximate Soft Shadows. In {\it Proceedings of the 2006 symposium on Interactive 3D graphics and games SI3D '06}, ACM Press pp. 63-69.

\end{thebibliography}


\end{document}
